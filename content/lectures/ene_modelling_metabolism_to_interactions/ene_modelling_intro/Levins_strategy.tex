\documentclass[xcolor=x11names,compress]{beamer}

%% General document %%%%%%%%%%%%%%%%%%%%%%%%%%%%%%%%%%
\usepackage{graphicx}
\graphicspath{{graphics/}}
\usepackage{tabularx}
\usepackage{booktabs}
\usepackage{tikz}
\usepackage{color,colortbl}
\usepackage{framed}
\usepackage{textcomp, setspace} %Needed for customization of ``listings''
% package
\usepackage[procnames]{listings} % to display code; don't forget [fragile]
% option after \begin{frame}
\input{inputs/rgb}
\definecolor{shadecolor}{rgb}{1,.9,.3}

\usepackage [autostyle]{csquotes}
\MakeOuterQuote{"}

\lstset{
    backgroundcolor=\color{shadecolor},
    tabsize=4,
    rulecolor=,
    language=python,
        basicstyle=\ttfamily\setstretch{1},
        upquote=true,
        aboveskip={1.5\baselineskip},
        columns=fixed,
        showstringspaces=false,
        extendedchars=true,
        breaklines=true,
        prebreak = \raisebox{0ex}[0ex][0ex]{\ensuremath{\hookleftarrow}},
        frame=single,
        showtabs=false,
        showspaces=false,
        showstringspaces=false,
        identifierstyle=\ttfamily,
        keywordstyle=\color[rgb]{0,0,1},
        commentstyle=\color[rgb]{0.133,0.545,0.133},
        stringstyle=\color[rgb]{0.627,0.126,0.941},
        numbers=left, 
        numberstyle=\tiny, 
        stepnumber=2, 
        numbersep=5pt
}

%%%%%%%%%%%%%%%%%%%%%%%%%%%%%%%%%%%%%%%%%%%%%%%%%%%%%%

%% Beamer Layout %%%%%%%%%%%%%%%%%%%%%%%%%%%%%%%%%%
\usetheme{Madrid}
\usecolortheme{seahorse}
\useoutertheme[subsection=false,shadow]{miniframes}
\useinnertheme{default}
% \usefonttheme{serif}
\usepackage{palatino}

\setbeamerfont{title like}{shape=\scshape}
\setbeamerfont{frametitle}{shape=\scshape, series = \bfseries}
\setbeamertemplate{frametitle}[default][center]
\setbeamertemplate{headline}{} %suppress headline (navigation pane)

\setbeamertemplate{footline}
{
  \leavevmode%
  \hbox{
  \begin{beamercolorbox}[wd=.1\paperwidth,ht=2.2ex,dp=1ex,center]{author in head/foot}
    \usebeamerfont{author in head/foot}\insertshortauthor
  \end{beamercolorbox}
  \begin{beamercolorbox}[wd=.8\paperwidth,ht=2.2ex,dp=1ex,center]{title in head/foot}%
    \usebeamerfont{title in head/foot}\insertshorttitle\hspace*{3em}
  \end{beamercolorbox}}
  \begin{beamercolorbox}[wd=.05\paperwidth,ht=2.2ex,dp=1ex,center]{}
     \insertframenumber{} / \inserttotalframenumber\hspace*{-3ex}
  \end{beamercolorbox}
}

\newcommand{\bcols}{\begin{columns}}
\newcommand{\ecols}{\end{columns}}
\newcommand{\bcol}[1]{\begin{column}{#1}}
\newcommand{\ecol}{\end{column}}

\def\signed #1{{\leavevmode\unskip\nobreak\hfil\penalty50\hskip2em
  \hbox{}\nobreak\hfil(#1)%
  \parfillskip=0pt \finalhyphendemerits=0 \endgraf}}

\newsavebox\mybox
\newenvironment{aquote}[1]
  {\savebox\mybox{#1}\begin{quote}}
  {\signed{\usebox\mybox}\end{quote}}
%%%%%%%%%%%%%%%%%%%%%%%%%%%%%%%%%%%%%%%%%%%%%%%%%%%%

\title{Understanding Model Trade-offs in Ecology and Evolution (Based on Levins (1966))}
\subtitle{\it A MulQuaBio Lecture}
% \author{}  
%%%%%%%%%%%%%%%%%%%%%%%%%%%%%%%%%%%%%%%%%%%%%%%%%%%%

\begin{document}

\frame{\titlepage}

\begin{frame}{Overview}
\begin{itemize}
    \item Models help simplify ecological and evolutionary complexity
    \item Levins (1966) identified a key trade-off:
    \item \textbf{Generality vs. Realism vs. Precision}
    \item You can’t maximize all three at once
    \item Let’s break these down with examples from ecology and evolution
\end{itemize}
\end{frame}

\begin{frame}{Generality}
\small
\begin{itemize}
    \item \textbf{What is it?} — How broadly the model applies across systems
    \item \textbf{Simple definition:} “Works for many species/situations”
    \item \textbf{Ecological example:} Logistic growth model
    \begin{itemize}
        \item \( \frac{dN}{dt} = rN(1 - \frac{N}{K}) \)
        \item Applies to many populations, regardless of species
        \item Ignores age, environment, interactions
    \end{itemize}
    \item \textbf{Evolutionary example:} Hardy--Weinberg model (null expectation)
    \begin{itemize}
        \item Genotype frequencies: $p^2$, $2pq$, $q^2$ (with $p + q = 1$)
        \item Applies broadly as a baseline across many diploid populations
        \item Ignores selection, drift, structure, mutation, migration
    \end{itemize}
\end{itemize}
\end{frame}

\begin{frame}{Realism}
\small
\begin{itemize}
    \item \textbf{What is it?} — How accurately the model reflects biological detail
    \item \textbf{Simple definition:} “Captures the messy real world”
    \item \textbf{Ecological example:} Spatially explicit individual-based predator--prey model
    \begin{itemize}
        \item Includes movement, behavior, stochasticity
        \item Can match a particular system well
        \item But highly specific and hard to generalize
    \end{itemize}
    \item \textbf{Evolutionary example:} Forward-time simulation with explicit genotypes and selection
    \begin{itemize}
        \item Tracks individuals, inheritance, recombination, and selection on traits
        \item Realistic mechanisms, but parameter-hungry
        \item Often used for “what if” scenarios rather than universal laws
    \end{itemize}
\end{itemize}
\end{frame}

\begin{frame}{Precision}
\small
\begin{itemize}
    \item \textbf{What is it?} — How exact or quantitative the model predictions are
    \item \textbf{Simple definition:} “Makes sharp forecasts”
    \item \textbf{Ecological example:} Short-term population forecast (state-space model)
    \begin{itemize}
        \item With good data, can produce tight prediction intervals for $N_{t+1}$
        \item Useful for management, but often system- and time-window-specific
    \end{itemize}
    \item \textbf{Evolutionary example:} Trait evolution under directional selection (quantitative genetics)
    \begin{itemize}
        \item Predicts a response like $R = h^2 S$ given heritability $h^2$ and selection differential $S$
        \item Can be very precise when assumptions hold and parameters are well estimated
        \item Precision does not guarantee accuracy if assumptions break (e.g., changing environments, non-additive genetics)
    \end{itemize}
\end{itemize}
\end{frame}

\begin{frame}{Realism vs. precision (and accuracy)}
\begin{itemize}
    \item \textbf{Realism} asks: \emph{Are the ecological and evolutionary mechanisms/assumptions biologically faithful?}
    \begin{itemize}
        \item High realism can still be wrong if key processes are missing or mis-specified.
    \end{itemize}

    \item \textbf{Precision} asks: \emph{Are the predictions tightly constrained (low uncertainty)?}
    \begin{itemize}
        \item High precision can still be misleading if it is consistently ``off''.
    \end{itemize}

    \item \textbf{Accuracy} asks: \emph{How close are predictions to the truth (on average)?}
    \begin{itemize}
        \item Intuition: \textbf{accuracy} depends on both \textbf{bias} (systematic error) and \textbf{variance} (scatter).
        \item A model can be \textbf{precise but inaccurate} (low variance, high bias) or \textbf{realistic but imprecise} (low bias, high variance).
    \end{itemize}
\end{itemize}
\end{frame}

\begin{frame}{Why trade-offs matter}
\begin{itemize}
    \item You must sacrifice one:
    \begin{itemize}
        \item Want realism + precision? Lose generality
        \item Want general + precise? Lose realism
        \item Want general + realistic? Lose precision
    \end{itemize}
    \item \textbf{There's no free lunch in modelling!}
\end{itemize}
\end{frame}

\begin{frame}{Summary table}
\centering
\footnotesize
\setlength{\tabcolsep}{4pt}
\renewcommand{\arraystretch}{1.25}
\begin{tabularx}{\textwidth}{@{}l>{\raggedright\arraybackslash}X>{\raggedright\arraybackslash}X>{\raggedright\arraybackslash}X@{}}
\toprule
\textbf{Goal} & \textbf{Simple definition} & \textbf{Example} & \textbf{Main trade-off} \\
\midrule
Generality & Applies to many systems & Ecology: logistic growth; evolution: Hardy--Weinberg & Ignores detail \\
Realism & Includes real-world complexity & Ecology: spatial IBM; evolution: forward simulation & Hard to generalize \\
Precision & Sharp numerical predictions & Ecology: short-term forecast; evolution: $R = h^2 S$ & Often assumption-sensitive \\
\bottomrule
\end{tabularx}
\end{frame}

\begin{frame}{Student discussion points}
\begin{itemize}
    \item Can you think of an example from your field where a model prioritizes one goal over the others?
    \item Are modern models (e.g., simulations, machine learning) changing this trade-off?
    \item When is it \emph{better} to be less realistic or less precise?
    \item Is generality always desirable in ecology and evolution?
    \item How might a model cluster approach help manage these trade-offs?
\end{itemize}
\end{frame}

\end{document}
